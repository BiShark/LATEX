\documentclass[11pt,twoside]{article}
\usepackage[T1, T2A]{fontenc}
\usepackage[utf8x]{inputenc}
\usepackage{mathtools} 
\usepackage[integrals]{wasysym}
%\usepackage{fancyhdr} 
%\pagestyle{fancy}
%\lhead{136}
%\rhead{\it\S\ \it {\sl{47}. Криволинейные интегралы}}
\makeatletter
\renewcommand{\@oddhead}{134\hfill\emph{\S\ {\sl47}. Криволинейные интегралы}}
\renewcommand{\@oddfoot}{}
\renewcommand{\@evenhead}{{\sl 47.7.} \emph {Геометрический смысл знака якобиана}\hfill 135}
\renewcommand{\@evenfoot}{}
\makeatother
\textwidth=122mm
\textheight=199mm
\newcommand\mes{\mathop{\mathrm{mes}}\nolimits}
\begin{document}
\noindent торой состоит из конечного числа кусучно-гладких кривых.\;Мы,\linebreak однако, не будем останавливаться на доказательстве этого факта,\linebreak а ограничимся лишь его формулировкой. При этом, используя опре-\linebreak деление 7, мы запишем формулу (47.19) в более компактном виде.

\textbf{\textit{Теорема} {\sl\textbf 1'}.} \emph{\!Пусть\ граница плоской ограниченной области\linebreak G состоит из конечного числа кучно-гладких кривых. Тогда,\linebreak если функции P, Q, $ \frac{\partial P} {\partial y} $ и $ \frac{\partial Q} {\partial x} $ непрерывны на $\overline{G}$, то
   $$\displaystyle\iint\limits_G \left( \frac{\partial Q} {\partial x} - \frac{\partial P} {\partial y} \right) dx\,dy = {\int\limits_{\partial\,G}} P\,dx + Q\,dy,$$
где ${\partial G}$ --- положительно ориентированная граница области G.}

\begin{center}

\setcounter{section}{47}
\setcounter{subsection}{6}
\subsection{Вычисление площадей с помощью\linebreak криволинейных интегралов}
\subsection{\bf 47.6 Вычисление площадей с помощью\linebreak}
\end{center}
\paragraph{криволинейных интегралов}




Пологая в формуле Грина $Q = x, P = 0$, получим
$$\iint\limits_G dx\,dy = \int\limits_{\gamma^+} xdy$$
и, следовательно,
$$\small\mes G = \int\limits_{\gamma^+}xdy.\eqno{(47.20)}$$
Аналогично, полагая P = --y, Q = 0, получим
$$\small\mes G = -\int\limits_{\gamma^+} ydx.\eqno{(47.21)}$$
Складывая формулы (47.20) и (47.21), будем иметь
$$\small\mes G = \frac12 \int\limits_{\gamma^+} xdy - ydx.\eqno{(47.22)}$$

    Найдем с помощью этой формулы в качестве примера пло-\linebreak щадь, ограниченную астроидой (см. в т. I рис. 61) $x = acos^3t,\linebreak y = asin^3t,\,{0\leq t\leq 2\pi.}$ Замечая, что здесь возрастание парамет-\linebreak ра $t$ соответствует положительной ориентации контура, имеем
\begin{split}
\par
$$ S = \frac12\int\limits_{\gamma^+} xdy - ydx = \frac{3a^2} {2} \int\limits_0^{2\pi} (cos^4 t sin^2 t + sin^4 t cos^2 t) dt = \\ $$
    $$ = {\frac{3a^2}{8}\int\limits_0^{2\pi} sin^2\,2t dt = \frac{3a^2} {16} \int\limits_0^{2\pi} (1 - cos 4t) dt = \frac{3\pi a^2} {8}.} $$
\par
\end{split}


\newpage
%\rhead{135}
%\lhead{47.7 \emph {Геометрический смысл знака якобиана}}

\begin{center}
\subsection{\bf 47.7 Геометрический смысл знака якобиана\linebreak отображения плоских областей}
\end{center}

Пусть $F$ --- взаимно однозначное непрерывно дифферен-\linebreak цируемое отображение плоской области $G\!\!\!\!\subset\!\!\!\!E^2_{uv}$ в плоскость $E^2_{xy}$\linebreak с якобианом, всюду в {\it G} не равным нулю. Тогда в силу принципа\linebreak сохранения области множество $G^* = F(\!G\!)$ также является областью\linebreak (см. п. 41.5), а якобиан в силу его непрерывности сохраняет знак на\linebreak $G$ (см. теорему 4 в п. 19.4), т.е. либо всюду на $G$ положителен, либо\linebreak всюду отрицателен.

В координатной записи отображение $F$ задаётся формулами
$$\left.
  \begin{array}{ccc}
    x & = & x\,(u, v), \\
    y & = & y\,(u, v), \\
  \end{array}
\right\} \eqno{(47.23)}$$  
причем, если $ M = (u,\,v),\,M^* = (x,\,y)$, то $M^* = F(M).$

Будем предпологать еще, что смешанные производные $\frac{\partial^2\:y}{\partial v\:\partial u}$\linebreak
и $\frac{\partial^2\:y}{\partial u\:\partial v}$ непрерывны, а следовательно, и равны друг другу во всех точках $G$.

Пусть теперь $\gamma$ - простой замкнутый кусочно-гладкий контур,\linebreak
расположенный в области $G$. Тогда (см. п. 46.2)\:$\gamma^* = F(\gamma)$ также яв-\linebreak
ляется простым замкнутым кусочно-гладким контуром. Пусть кон-\linebreak
тур $\gamma$ является границей ограниченной области $\Gamma\! \subset \!G^*$, а контур\linebreak
$\gamma^*$ - ограниченной области $\Gamma^*\subset G^*$. Пусть $\Gamma^*\subset F(\Gamma)$ и области $\Gamma$\linebreak
и $\Gamma^*$ таковы, что к ним применима формула Грина, например, они\linebreak
удолетворяют условиям, налагаемым на область в теореме 1.\linebreak
(На самом деле, как это уже отмечалось, при сделанных предположе-\linebreak
ниях это всегда имеет место, однако это не было доказано.)

Пусть, наконец,
$$\begin{array}{ccc}
    u = u\,(t), \\
      v = v\,(t), \\
      a\leq t\leq b, 
\end{array}$$
\begin{flushleft}
\item[$-$] \noindent представление контура $\gamma^+$, и, следовательно,
\end{flushleft}
$$\begin{array}{ccc}
       x = x[u\,(t), v\,(t)], \\
         y = y[u\,(t), v\,(t)], \\
         a\leq t\leq b, 
\end{array} \eqno{(47.24)}$$
\begin{flushleft}
\item[$-$] \noindent некоторое представление контура $\gamma^*$.
\end{flushleft}




\newpage
\end{document}
